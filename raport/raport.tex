\newcommand{\documentfaculty}{Wydział Fizyki i Astronomii\\kierunek Informatyka stosowana i systemy pomiarowe\\II rok}
\newcommand{\documentauthor}{Ewa Śnieżyk-Milczyńska}
\newcommand{\documentsubject}{Projekt C++}
\newcommand{\documenttitle}{Założenia projektu}
\newcommand{\documentdate}{Data: 16 października 2023}

\makeatletter
\def\input@path{{../../../latex/}}
\makeatother

\documentclass[12pt,twoside,a4paper]{article}

\usepackage{polski}
\selecthyphenation{polish}
\usepackage[utf8]{inputenc}

\usepackage[pdftex]{hyperref}
\hypersetup{
   pdftitle = {\documenttitle},
   pdfauthor = {\documentauthor}
}

\pagestyle{plain}

\begin{document}
\thispagestyle{empty}

\input{ewaheader.tex}

\section{Wstęp}
Autor zamierza napisać grę w~warcaby w~języku C++. Aplikacja będzie skompilowana pod system operacyjny Linux.

\section{Język programowania}

C++ to język programowania, który jest kompatybilny z językiem C i zawiera dodatkowe funkcje i możliwości.
% Język C++ został opracowany w latach 80. XX w. przez Bjarne Stroustrupa i jest szeroko używany w programowaniu komputerów,
% w tworzeniu oprogramowania, a także w rozwoju aplikacji i gier.
% C++ pozwala tworzyć zarówno procedury i funkcje, jak i obiekty i klasy.
% Składnia C++ jest podobna do składni języka C, została jednak wzbogacona o nowe elementy, takie jak programowanie obiektowe, pojęcia klas i dziedziczenia.
% C++ umożliwia korzystanie z wskaźników, co pozwala na bardziej zaawansowane operacje na pamięci i manipulację danymi w niższym poziomie.
% C++ zawiera mechanizm szablonów, który pozwala na tworzenie ogólnych struktur danych i algorytmów, które mogą być dostosowywane do różnych typów danych.
% C++ jest wyposażony w bogatą bibliotekę standardową, która zawiera gotowe struktury danych i funkcje do wielu zastosowań, takie jak obsługa plików, operacje na łańcuchach znaków, obsługa wejścia/wyjścia.
% C++ jest używany w wielu dziedzinach, ponieważ jest to jeden z popularniejszych języków programowania na świecie, ze względu na swoją wszechstronność i wydajność.

\section{Zewnętrzna biblioteka}

Autor zamierza wykorzystać bibliotekę QT do stworzenia aplikacji do gry w warcaby.
Biblioteka Qt jest jednym z najpopularniejszych i najbardziej wszechstronnych zestawów narzędzi do tworzenia oprogramowania.
Została stworzona w celu ułatwienia tworzenia oprogramowania, z naciskiem na aplikacje graficzne i interfejsy użytkownika. 
Qt umożliwia pisanie oprogramowania, które działa na wielu platformach. Dzięki temu programiści mogą tworzyć jedną wersję swojej aplikacji, która działa na różnych systemach operacyjnych.
Qt zawiera narzędzia do tworzenia interfejsów użytkownika (GUI) za pomocą deklaratywnego języka QML
lub tradycyjnego narzędzia do projektowania interfejsu użytkownika. 
Qt dostarcza obszerną bibliotekę modułów, które ułatwiają tworzenie różnych rodzajów aplikacji, w~tym aplikacji desktopowych, mobilnych, gier, narzędzi programistycznych i~innych.
Qt obsługuje tworzenie aplikacji na wiele rodzajów urządzeń, w~tym smartfony, tablety, komputery, urządzenia wbudowane, roboty i inne.
Qt została stworzona przez norweskie przedsiębiorstwo Trolltech, a obecnie jest rozwijany przez firmę The Qt Company. 

\section{Gra w warcaby}
Gra w~warcaby to planszowa gra strategiczna dla dwóch graczy, którzy siedzą naprzeciwko siebie. 
Plansza do gry w~warcaby ma 100 pól ułożonych w~siatkę $10\times 10$.
Każdy gracz na początku gry ma 20 pionów (białych lub czarnych, w~zależności od koloru gracza).
Gra polega na eliminowaniu pionów przeciwnika i staraniu się zdobyć przewagę.

\section{Zasady gry}

Gra w~warcaby odbywa się w~turach, a gracze wykonują swoje ruchy na przemian.
Białe piony poruszają się po planszy w~kierunku przeciwnika, a~czarne w~kierunku gracza.
Piony przemieszczają się na przemian o jedno pole na skos w~lewo lub w~prawo.
Jeśli pion dojdzie do ostatniego rzędu przeciwnika, zostaje awansowany na damkę.
Ruchy damek są bardziej złożone. Mogą poruszać się w~pionie lub poziomie o~dowolną liczbę pól,
o~ile droga jest wolna.
Damka może zbijać przeciwnika, poruszając się na skos w~przód lub w~tył.
Aby zbić piona przeciwnika, należy przeskoczyć go na planszy po skosie.
Po zbiciu pion przeciwnika jest usuwany z~planszy.
Jeśli istnieje możliwość kolejnego bicia (tzw. bicie wielokrotne), gracz jest zobowiązany je wykonać.
Celem gry jest zbicie wszystkich pionów przeciwnika lub uniemożliwienie mu wykonywania ruchów.
Strategia, planowanie, oraz zdolność do przewidywania ruchów przeciwnika są kluczowe dla sukcesu w~tej grze.

\section{Założenia projektu}

Autor zamierza napisać rozgrywkę uproszczoną, bez przechodzenia do etapu damek. Gracz będzie grał z komputerem.

\section{Etapy pracy}

\begin{enumerate}
   \item Wyświetlanie ekranu powitalnego z~wyborem koloru pionów.
	\item Wyświetlanie planszy i pionów.
	\item Przesuwanie pionów gracza po planszy za pomocą myszki. Zaprogramowanie dozwolonych ruchów. Na tym etapie przeciwnik nie wykonuje ruchów.
   \item Po ruchu gracza przeciwnik (komputer) wykonuje losowy dozwolony ruch. Ruchy wykonuje się na przemian.
   \item Opracowanie algorytmu opcjonalnego bicia. Modyfikacja funkcji sprawdzającej dozwolone ruchy.
   \item Wprowadzenie reguły obowiązkowego bicia jednokrotnego.
   \item Wprowadzenie reguły obowiązkowego bicia wielokrotnego.
   \item Wykrywanie końca gry i ogłaszanie zwycięzcy.
   \item Udoskonalenie algorytmu gry przeciwnika (komputera), aby potrafił wybierać ruchy najbardziej korzystne dla niego, np. bicie wielokrotne.
\end{enumerate}

\section{Źródła}

Autor będzie używał słownictwa i zasad gry według książki Marii Moldenhawer-Frej i Romualda Freja pt. Dama królową gier. Warcaby stupolowe. Wyd. Instytut Wydawniczy Związków Zawodowych Warszawa 1989.
\end{document}

